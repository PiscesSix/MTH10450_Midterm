\documentclass[12pt]{article}
\usepackage[margin=1in]{geometry} 
\usepackage[utf8]{vietnam}
\usepackage{amsmath}
\usepackage{tcolorbox}
\usepackage{amssymb}
\usepackage{amsthm}
\usepackage{lastpage}
\usepackage{fancyhdr}
\usepackage{accents}
\usepackage{bbm}
\pagestyle{fancy}
\setlength{\headheight}{40pt}

\begin{document}

\lhead{Nguyen Minh Tri $-$ 21110417} 
\rhead{Thuat toan toi uu} 
\cfoot{\thepage\ of \pageref{LastPage}}

\begin{tcolorbox}
\textbf{Các chứng minh trên lớp có đề cập:} (riêng phần nghiệm tối ưu thầy lướt hơi nhanh nên không để ý @@)
\begin{enumerate}
	\item Chứng minh đạo hàm theo hướng, mệnh đề $1.2$ (Trang $14$), quyển 1
	\item Chứng minh nếu $f$ là hàm afin $(f(x) = \langle c,x \rangle + \alpha)$ thì:
	$$
		\forall x, y \in \mathbb{R}^n, \forall \lambda, \mu \in \mathbb{R} \text{ mà } \lambda + \mu = 1 \Rightarrow f(\lambda x + \mu y) = \lambda f(x) + \mu f(y)
	$$
	Chứng minh:
		\begin{align*}
			f(\lambda x + \mu y) &= \langle c, \lambda x + \mu y\rangle + \alpha \\
								 &= \langle c, \lambda x\rangle + \langle c, \mu y \rangle + \alpha \\
								 &= \lambda \langle c,x\rangle  + \mu \langle c,y \rangle + (\lambda + \mu)\alpha  \\
								 &= \lambda \langle c,x\rangle + \lambda\alpha + \mu \langle c,y \rangle + \mu\alpha \\
								 &= \lambda f(x) + \mu f(y)
		\end{align*}
\end{enumerate}
\end{tcolorbox}
\section{Lý thuyết + Bài tập}
\begin{enumerate}
	\item Nêu định nghĩa tập afine, lồi, hàm lồi và các tính chất cơ bản
	\begin{itemize}
		\item Tập afin: Nếu $M$ chứa trọn cả \textbf{đường thẳng} đi qua hai điểm bất kì thuộc $M$, tức là:
		$$
			\forall x_1, x_2 \in M, \; \lambda \in \mathbb{R} \Rightarrow \lambda x_1 + (1-\lambda)x_2 \in M
		$$
		\begin{itemize}
			\item Tổ hợp afin:
			$$
				x = \sum_{i=1}^k \lambda_i x_i \quad \lambda_1,\dots,\lambda_k \in \mathbb{R} \text{ và } \sum_{i=1}^k \lambda_i = 1
			$$
			\item Bao afin của $E$: Giao của tất cả các tập afin chứa $E$, cũng là tập afin nhỏ nhất chứa $E$, ký hiệu aff$E$
		\end{itemize}				
		
		\item Tập lồi: Nếu nó chứa trọn \textbf{đoạn thẳng} nối hai điểm bất kỳ thuộc nó, tức:
		$$
			\forall x_1, x_2 \in M, \; 0 \leq \lambda  \leq 1: \; \lambda x_1 + (1-\lambda)x_2 \in M
		$$
		\begin{itemize}
			\item Nếu $M$ là tập lồi thì $\alpha M$ cũng là tập lồi.
			\item Nếu $M_1, M_2$ là hai tập lồi thì $M_1 + M_2$ cũng là tập lồi.
			\item Một tập là tập lồi khi và chỉ khi nó chứa tất cả các tổ hợp lồi của những phần tử thuộc nó.
			\item Tổ hợp lồi:
			$$
				x = \sum_{i=1}^k \lambda_i x_i \quad \lambda_1,\dots,\lambda_k \geq 0 \text{ và } \sum_{i=1}^k \lambda_i = 1
			$$
		\end{itemize}				
		
		\item Hàm lồi: Hàm $f$ được xác định trên tập $X \subset \mathbb{R}^n:$
		$$
			f(\lambda x_1 + (1-\lambda)x_2) \leq \lambda f(x_1) + (1-\lambda)f(x_2)
		$$
		\item Các tính chất của hàm lồi:
		\begin{enumerate}
			\item Hàm lồi thì epif là tập lồi 
			\item $f$ là hàm lồi thì tập mức dưới $L_\alpha (f)$ cũng là tập lồi với mọi $\alpha \in \mathbb{R}$
			\item $f'(x_0,d) \leq f(x_0 +d) - f(x_0)$
			\item $f$ khả vi trên tập lồi mở:
			$$
				f(y) - f(x) \geq \langle \nabla f(x), y-x\rangle
			$$
			\item Ma trận Hesse nửa xác định dương (xác định dương thì bỏ dấu $=$):
			$$
				y^T \nabla^2 f(x) y \geq 0 \quad \forall y \in \mathbb{R}^n
			$$
			\item Cho hàm toàn phương:
			$$
				f(x) = \frac{1}{2} \langle x, Qx \rangle + \langle x,a \rangle + \alpha
			$$
			khi đó, $f$ là hàm lồi khi $Q$ là ma trận nửa xác định dương.
		\end{enumerate}
	\end{itemize}
	\item[]
		\begin{tcolorbox}
			Một số bài tập: bài 1, 2, 3, 13, 14, 15, \textbf{16, 17}, 19, 20 (Trang 36 $-$ 38).
		\end{tcolorbox}
	\item Với dữ kiện đã cho, phát biểu mô hình bài toán tối ưu (Trong sách có đề cập, chương 2, trang 39 $-$ 50).
	\item Phát biểu điều kiện cần và điền kiện đủ của sự tồn tại điểm cực tiểu của bài toán khả vi không ràng buộc.
	\begin{itemize}
		\item Điều kiện cần: Cho $f$ xác định, khả vi trên $\mathbb{R}^n$. Nếu $x^* \in \mathbb{R}^n$ là nghiệm cực tiểu địa phương của bài toán ($P^{krb}$) thì $\nabla f(x^*) = 0.$
		\item Điều kiện đủ: Giả sử $f$ khả vi liên tục hai lần trên $\mathbb{R}^n.$ Khi đó:
		\begin{itemize}
			\item Nếu $x^* \in \mathbb{R}^n$ là điểm cực tiểu địa phương của $f$ trên $\mathbb{R}^n$ khi:
			\begin{align*}
				&\nabla f(x^*) = 0 \\
				&\nabla^2 f(x^*) \text{ nửa xác định dương}	
			\end{align*}
			\item Trong trường hợp $\nabla^2 f(x^*)$ xác định dương thì $x^*$ là điểm cực tiểu địa phương chặt của $f$.
		\end{itemize}
	\end{itemize}
	\item Trình bày thuật toán gradient với thủ tục tìm chính xác theo tia và thủ tục quay lùi. (Xem lại các bước trong sách trang 231, quyển 2).
	\item[]
		\begin{tcolorbox}
			Một số bài tập: bài 5, 6, \textbf{7, 11} (Trang 289 $-$ 291).
		\end{tcolorbox}
\end{enumerate}






\end{document}